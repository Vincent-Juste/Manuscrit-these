The primary goal of this PhD work was to develop an analysis of the single detector triggers for the MBTA pipeline.
There were two main challenges to face.
The first one, discussed in chapter \ref{section:selection}, was to reduce the background in order to have the best possible separation between astrophysical signals and background in terms of ranking statistics.
The second one, detailed in chapter \ref{section:far} was to estimate the significance of the candidates through the computation of a false alarm rate.\\

To reduce the background, the first decision was to focus on a smaller region of the parameter space.
This region, called EM bright and defined as hasRemnant $> 0.1\%$, is associated to longer templates and encompasses the triggers that have a chance to have an electromagnetic counterpart.
The background associated to this EM bright population was already much lower, both in terms of number of trigger and rwSNR values.
In order to prepare future studies, we still considered the EM dark population (triggers complementary to the EM bright population).
This population is associated to much shorter templates.
It was shown that some short templates produce more triggers with sometimes high ranking statistics.
We then used noise rejection tools to define selection criteria for the single detector triggers.
These criteria allow to significantly reduce the background and have a clearer separation between astrophysical signals and noise triggers.
The losses of duty cycle due to the criteria are 9.1\% in H1, 15.4\% in L1 and 9.8\% in V1.
They are largely compensated by the gain in volume for the search.\\

Once the background was under control, we looked into ways to compute the false alarm rate of the candidates.
Since the background has an exponential shape after we apply the selection criteria, a straightforward way is to parametrize the rwSNR distribution with an exponential function.
However this method is not guaranteed to be robust in case of some bad days with lots of noise.
We therefore considered a different approach.
Motivated by the method used for the coincidence search, we use MBTA's 2-band structure to build an estimation of the background by making fake coincidences with the signals observed in each band.
% These fake coincidences are made by combining individual band triggers with individual band random noise or triggers at random times.
% The only requirement to combine them is that their real templates can be combined into a virtual template of the bank.
% We used this method to make two estimations of the background.
% One using only trigger-random noise coincidences and the other using trigger-trigger coincidences.
% The estimation obtained from trigger-random noise coincidences generally under-estimated the background.
% On the other hand the one obtained from trigger-trigger coincidences largely over-estimates the background and has a less steep slope.
% To avoid sending false alerts at the begining of O4 it was decided to use the over-estimated background, scaled to have a rate comparable to the observed background.
The computed background also has an exponential shape and therefore validate the extrapolation which could be made on the observed background distribution to reach low false alarm rate values.\\

Two additional studies were conducted to pave the way for possible future improvements of the pipeline.
One was an attempt at improving a signal consistency test to recover more astrophysical signals and make them more significant, described in chapter \ref{section:rwSnr}.
A known issue was that this signal consistency test tends to downgrade too heavily very loud events.
This means that loud astrophysical events can become much less significant after application of the test.
This behaviour comes from the discrete nature of the template bank.
To mitigate this effect a modified test was defined by including a term dependent on the SNR.
It allows to ``flatten'' the behaviour of the test result as a function of the SNR.
This was followed-up with some work to tune the parameters of the reweighting formula used to downgrade the candidates in order to maximize the recovery of injections while not increasing the background.
The results were not convincing enough to decide to proceed with the changes for the fourth observing run but it could be used by future studies.\\

The other work carried out in section \ref{section:bad_triggers} focused on understanding the differences between single detector triggers obtained from real data and single detector triggers obtained by analyzing simulated Gaussian noise (strain).
We showed that single detector triggers tend to be detected at times where there was a drop in the detector sensitivity, meaning that their SNR is over-estimated.
We then explored the possibility of using the detector range variations to apply a correction to the SNR in order to mitigate the effect.
Finally we also highlighted differences between the range computed on real data and the range computed on Gaussian noise.\\

All of this work allowed to start safely the single detector triggers search at the begining of O4.
There is still room for improvements in all aspects to make the search even more sensitive during O4 as we gradually increase our knowledge of the background.